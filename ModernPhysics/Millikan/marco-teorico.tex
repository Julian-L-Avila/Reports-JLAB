\documentclass[journal, table]{IEEEtran}
\usepackage[utf8]{inputenc}
\usepackage[english, spanish]{babel}
\usepackage[sorting=none]{biblatex} 
\bibliography{ref.bib}

\usepackage{amsmath, amsfonts, amsthm, amssymb, mathtools}
\usepackage{hyperref, url}
\usepackage[dvipsnames]{xcolor}
\usepackage{fancyhdr, last page}
\usepackage{siunitx}
\usepackage{graphicx}
\usepackage[affil-sl]{authblk}
\usepackage{anyfontsize}
\usepackage{graphicx}
\usepackage{csquotes}
\usepackage{float}
\usepackage{svg}
\usepackage{tabularx, ragged2e, booktabs}
\usepackage{siunitx}
\usepackage[spanish]{cleveref}
\usepackage{multirow}
\usepackage{adjustbox}
\AtBeginDocument{\decimalpoint}
\usepackage{caption}
\usepackage{subcaption}

\setkeys{Gin}{width=0.8\linewidth}
\hypersetup{
    colorlinks=true,
    linkcolor=black,
    urlcolor=blue,
    pdftitle={Lab - JLBA}
}
\urlstyle{same}
\sisetup{separate-uncertainty}

\begin{document}

\title{\textbf{} \\
    \small{}}

\author[*]{Laura Herrera
    \thanks{Laura Herrera: 20212107011}}
\author[*]{Bryan Martínez
    \thanks{Bryan Martínez: 20212107008}}
\author[*]{Julian Avila
    \thanks{Julian Avila: 20212107030}}
\author[*]{Juan Acuña
    \thanks{Juan Acuña: 20212107034}}


\affil[*]{Proyecto Curricular de Fisica\\ Universidad Distrital Francisco José de Caldas}

\markboth{}
{Shell \MakeLowercase{\textit{et al.}}: Bare Demo of IEEEtran.cls for IEEE Journals}

\maketitle

\selectlanguage{english}
\begin{abstract}

\end{abstract}

\begin{IEEEkeywords}

\end{IEEEkeywords}

\selectlanguage{spanish}
\begin{abstract}

\end{abstract}

\begin{IEEEkeywords}

\end{IEEEkeywords}

\section{Objetivos}
\begin{itemize}
    \item Calcular la carga del electrón
    \item Determinar la velocidad de subida y bajada de las gotas de aceite
    \item Determinar la carga y radio de una gota de aceite ionizada
\end{itemize}

\section{Marco teórico}

Entre las fuerzas ejercidas a la gota de aceite se encuentran la fuerza gravitacional $\Vec{F}_g$, la fuerza asociada al empuje o flotabilidad descrito por el principio de Arquimedes $\Vec{F}_E$ y la fuerza de fricción generada por el aire $\Vec{F}_d$ descrito por la ley de Stokes.\\

\begin{figure}
    \centering
    \includegraphics[width=0.5\linewidth]{}
    \caption{Diagrama de cuerpo libre de la gota de aceite en el primer intervalo de tiempo}
    \label{fig:enter-label}
\end{figure}

Inicialmente la gota es acelerada en dirección a la tierra pero prontamente llega a una velocidad terminal, esto indica un equilibrio de fuerzas.

\begin{equation}
    \begin{split}
        \sum \Vec{F}=0\\
        \lvert \Vec{F}_d\rvert + \lvert\Vec{F}_E\rvert + \lvert \Vec{F}_g\rvert =0
    \end{split}
\end{equation}

La fuerza de la gravedad, flotación y fricción se describe respectivamente de la siguiente manera en donde $V_{ac}$ representa el volumen de la gota de aceite y $\mu$ es el coeficiente de viscosidad del aire:

\begin{equation}
    \begin{split}
        \Vec{F}_g=m\Vec{g}\\
        \Vec{F}_E=-\rho_a \Vec{g}V_{ac}\\
        \Vec{F}_d=-6\pi r\mu \Vec{v}_c
    \end{split}
\end{equation}

Tomando en cuenta únicamente las magnitudes y remplazando se encuentra que:

\begin{equation}
    6\pi r\mu v_c=mg-\rho_a \Vec{g}V_{ac}
\end{equation}

Debido a que no se mide la masa de la gota, se dejará el volumen $V_{ac}$ y la masa $m$ en términos de su radio $r$ utilizando las ecuaciones

\begin{equation}
    \begin{split}
        V_{ac}=\frac{4}{3}\pi r^3\\
        m=\frac{4}{3}\pi r^3\rho_{ac}
    \end{split}
\end{equation}

Remplazando y operando se obtiene

\begin{equation}
    \begin{split}
        6\pi r\mu v_c=\frac{4}{3}\pi r^3(\rho_{ac}-\rho_a)g
    \end{split}
\end{equation}

de la cual se obtiene el valor para $r$

\begin{equation}
    \begin{split}
        r=\sqrt{\frac{9\mu v_c}{2g(\rho_{ac}-\rho_a)}}
    \end{split}
\end{equation}

Para el segundo intervalo de tiempo, se aplica una diferencia de potencial y la gota empieza a ascender hacia la placa superior, nuevamente al llegar a la velocidad terminal, esta vez llamada $v_a$ ocurre un equilibrio de fuerzas en donde $\Vec{F_e}$ es la fuerza eléctrica.

\begin{figure}
    \centering
    \includegraphics[width=0.5\linewidth]{}
    \caption{Diagrama de cuerpo libre de la gota de aceite en el segundo intervalo de tiempo}
    \label{fig:enter-label}
\end{figure}

Las ecuaciones para el diagrama de cuerpo libre se ve tal que

\begin{equation}
    \begin{split}
        \lvert\Vec{F}_E\rvert + \lvert \Vec{F}_e\rvert + \lvert \Vec{F}_g\rvert+\lvert \Vec{F}_d\rvert =0
    \end{split}
\end{equation}

Nuevamente se utilizan solo sus magnitudes, remplazando $F_e=qE$ donde $E$ es el campo eléctrico y operando se obtiene que

\begin{equation}
    \begin{split}
        qE=6\pi r\mu v_a + \frac{4}{3}\pi r^3(\rho_{ac}-\rho_a)
    \end{split}
\end{equation}

utilizando la ecuación que relaciona la fuerza de fricción con la fuerza de la gravedad y de flotabilidad, además de remplazar el campo electrico por su expresión en un capacitor $E=\frac{V}{d}$ se llega a que la expresión para la carga en la gota es:

\begin{equation}
    \begin{split}
        q=\frac{6\pi \mu d}{V}\sqrt{\frac{9\mu v_c}{2g(\rho_{ac}-\rho_a)}}(v_a-v_c)
    \end{split}
\end{equation}i

Por último, sabemos que la carga está cuantizada y debe ser múltiplo de un valor $e^-$ en dónde $n$ es el número de electrones en la gota y se cumple la relación

\begin{equation}
    q=ne^-
\end{equation}

Realizando suficientes medidas de q y sacando su mínimo común divisor, se puede llegar al valor de $e^-$

\section{Resultados y Análisis}


\section{Conclusiones}

\printbibliography
\end{document}

