\selectlanguage{english}
\begin{abstract}
The Millikan experiment is proposed to determine the charge of the electron. To achieve this, the falling velocity of an oil drop is measured under the influence of gravity, buoyant force, and friction. Subsequently, a potential difference is applied between plates, causing the drop to rise, and this ascent velocity is measured. The software \texttt{Tracker} was used in this experiment to determine these velocities.

Once the velocities are measured, an analysis of the involved forces is performed to calculate the radius and charge of each drop. A total of 20 drops were recorded, and using a code in \texttt{Julia}, the greatest common divisor of the obtained charges was determined. Finally, the charge of the electron was estimated to be approximately $e^- \approx \qty{1.50 \times 10^{-19}}{C}$.

\end{abstract}

\begin{IEEEkeywords}
Millikan Experiment, Velocities, Charge, Electron, Radius.
\end{IEEEkeywords}

\selectlanguage{spanish}
\begin{abstract}
Se propone realizar el experimento de Millikan para determinar la carga del electrón. Para ello, se mide la velocidad de caída de una gota de aceite bajo la influencia de la gravedad, la fuerza de empuje y la fricción. Posteriormente, se aplica una diferencia de potencial entre placas, lo que provoca una velocidad de ascenso de la gota, la cual se mide. En esta práctica se utilizó el software \texttt{Tracker} para determinar  dichas velocidades.

Una vez medidas las velocidades, se realiza un análisis de las fuerzas involucradas para calcular el radio y la carga de cada gota. Se registraron un total de 20 gotas, y mediante un código en \texttt{Julia} se determinó el mínimo común divisor de las cargas obtenidas. Finalmente, se estimó que la carga del electrón es aproximadamente $e^- \approx \qty{1.50 \times 10^{-19}}{C}$.

\end{abstract}

\begin{IEEEkeywords}
Experimento de Millikan, Velocidades, Carga, Electrón, Radio.
\end{IEEEkeywords}
