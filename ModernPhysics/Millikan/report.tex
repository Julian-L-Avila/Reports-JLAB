\documentclass[journal, table]{IEEEtran}
\usepackage[utf8]{inputenc}
\usepackage[english, spanish]{babel}
\usepackage[sorting=none]{biblatex} 
\bibliography{ref.bib}

\usepackage{amsmath, amsfonts, amsthm, amssymb, mathtools}
\usepackage{hyperref, url}
\usepackage[dvipsnames]{xcolor}
\usepackage{fancyhdr, last page}
\usepackage{siunitx}
\usepackage{graphicx}
\usepackage[affil-sl]{authblk}
\usepackage{anyfontsize}
\usepackage{graphicx}
\usepackage{csquotes}
\usepackage{float}
\usepackage{svg}
\usepackage{tabularx, ragged2e, booktabs}
\usepackage{siunitx}
\usepackage[spanish]{cleveref}
\usepackage{multirow}
\usepackage{adjustbox}
\AtBeginDocument{\decimalpoint}
\usepackage{caption}
\usepackage{subcaption}

\setkeys{Gin}{width=0.8\linewidth}
\hypersetup{
    colorlinks=true,
    linkcolor=black,
    urlcolor=blue,
    pdftitle={Lab - JLBA}
}
\urlstyle{same}
\sisetup{separate-uncertainty}

\begin{document}

\title{\textbf{} \\
    \small{}}

\author[*]{Laura Herrera
    \thanks{Laura Herrera: 20212107011}}
\author[*]{Bryan Martínez
    \thanks{Bryan Martínez: 20212107008}}
\author[*]{Julian Avila
    \thanks{Julian Avila: 20212107030}}
\author[*]{Juan Acuña
    \thanks{Juan Acuña: 20212107034}}


\affil[*]{Proyecto Curricular de Fisica\\ Universidad Distrital Francisco José de Caldas}

\markboth{}
{Shell \MakeLowercase{\textit{et al.}}: Bare Demo of IEEEtran.cls for IEEE Journals}

\maketitle

\selectlanguage{english}
\begin{abstract}

\end{abstract}

\begin{IEEEkeywords}

\end{IEEEkeywords}

\selectlanguage{spanish}
\begin{abstract}

\end{abstract}

\begin{IEEEkeywords}

\end{IEEEkeywords}

\section{Objetivos}
\begin{itemize}
    \item Calcular la carga del electrón
    \item Determinar la velocidad de subida y bajada de las gotas de aceite
    \item Determinar la carga y radio de una gota de aceite ionizada
\end{itemize}

\section{Marco teórico}
Entre las fuerzas ejercidas sobre la gota de aceite se encuentran la fuerza
gravitacional $\Vec{F}_g$, la fuerza asociada al empuje o flotabilidad descrito
por el principio de Arquímedes $\Vec{F}_E$ y la fuerza de fricción generada
por el aire $\Vec{F}_d$ descrito por la ley de Stokes.

\begin{figure}[htbp!]
    \centering
    %\includegraphics[width=0.8\linewidth]{}
    \caption{Diagrama de cuerpo libre de la gota de aceite en descenso}
    \label{fig:fbd-falling-drop}
\end{figure}

Inicialmente la gota es acelerada en dirección a la tierra pero prontamente
llega a una velocidad terminal, esto indica un equilibrio de fuerzas.
\begin{equation}
    \vec{F}_d + \vec{F}_E + \vec{F}_g = 0
\end{equation}

La fuerza de la gravedad, flotación y fricción se describe respectivamente de
la siguiente manera en donde $V_{ac}$ representa el volumen de la gota de
aceite y $\mu$ es el coeficiente de viscosidad del aire:
\begin{align}
    \vec{F}_g &= m\vec{g} \\
    \vec{F}_E &= -\rho_a \vec{g}V_{ac} \\
    \vec{F}_d &= -6\pi r\mu \vec{v}_c
\end{align}

Tomando en cuenta únicamente las magnitudes y remplazando se encuentra que:
\begin{equation}
    6\pi r\mu v_c = mg-\rho_a g V_{ac}
\end{equation}

Debido a que no se mide la masa de la gota, se dejará el volumen $V_{ac}$ y la
masa $m$ en términos de su radio $r$ utilizando las ecuaciones
\begin{align}
    V_{ac} = \frac{4}{3}\pi r^3\\
    m = \frac{4}{3}\pi r^3\rho_{ac}
\end{align}

Remplazando y operando se obtiene
\begin{equation}
    6\pi r\mu v_c = \frac{4}{3}\pi r^3(\rho_{ac}-\rho_a)g
\end{equation}

de la cual se obtiene el valor para $r$
\begin{equation}
    r = \sqrt{\frac{9\mu v_c}{2g(\rho_{ac}-\rho_a)}}
\end{equation}

Para el segundo intervalo de tiempo, se aplica una diferencia de potencial y la
gota empieza a ascender hacia la placa superior, nuevamente al llegar a la
velocidad terminal, esta vez llamada $v_a$ ocurre un equilibrio de fuerzas en
donde $\vec{F_e}$ es la fuerza eléctrica.
\begin{figure}
    \centering
    %\includegraphics[width=0.5\linewidth]{}
    \caption{Diagrama de cuerpo libre de la gota de aceite en ascenso}
    \label{fig:fbd-ascending-drop}
\end{figure}

Las ecuaciones para el diagrama de cuerpo libre se ve tal que
\begin{equation}
    \vec{F}_E + \vec{F}_e + \vec{F}_g + \vec{F}_d = 0
\end{equation}

Nuevamente se utilizan solo sus magnitudes, remplazando $F_e=qE$ donde $E$ es
el campo eléctrico y operando se obtiene que
\begin{equation}
    qE = 6\pi r\mu v_a + \frac{4}{3}\pi r^3(\rho_{ac}-\rho_a)
\end{equation}

utilizando la ecuación que relaciona la fuerza de fricción con la fuerza de la
gravedad y de flotabilidad, además de remplazar el campo eléctrico por su
expresión en un capacitor $E=\frac{V}{d}$ se llega a que la expresión para la
carga en la gota es:
\begin{equation}
    q = \frac{6\pi \mu d}{V}\sqrt{\frac{9\mu v_c}{2g(\rho_{ac}-\rho_a)}}(v_a-v_c)
\end{equation}

Por último, sabemos que la carga está cuantizada y debe ser múltiplo de un
valor $e^-$ en dónde $n$ es el número de electrones en la gota y se cumple la
relación
\begin{equation}
    q=ne^-
\end{equation}

Realizando suficientes medidas de $q$ y sacando su mínimo común divisor,
se puede llegar al valor de $e^-$


\section{Materiales y métodos}
Parte de materiales y metodos


\section{Resultados y Análisis}

\section{Conclusiones}

\printbibliography
\end{document}
