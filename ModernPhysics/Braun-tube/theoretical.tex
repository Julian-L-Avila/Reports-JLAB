Para determinar la carga específica del electrón, en primera instancia,
acudir a la energía cinética  en dónde $K$ es la energía cinética del electrón
al ser acelerado, por ello, existen dos expresiones para la misma.
\begin{equation}\label{eq:kvx}
  \begin{split}
    K=\frac{1}{2}m_ev_x^2\\
    K=eU_a
  \end{split}
\end{equation}

Igualando las ecuaciones de \ref{eq:kvx} se llega a la expresión
\begin{equation}\label{eq:k=k}
  eU_a=\frac{1}{2}m_ev_x^2
\end{equation}

Se puede hallar la velocidad de salida de los electrones $v_x$ despejando la
ecuación \ref{eq:k=k}
\begin{equation}\label{v=sqrt}
  v_x=\sqrt{\frac{2eU_a}{m_e}}
\end{equation}

Se activa un campo magnético que equilibra la deflexión del haz de electrones,
como el campo magnético de inducción  y la velocidad de los electrones están en
direcciones perpendiculares entre sí.
La condición para la velocidad es
\begin{equation}\label{v=E/B}
  v_x=\frac{E}{B}
\end{equation}

En donde el valor del campo magnético $B$ en el eje axial  de las bobinas de
Helmholtz está dado por \cite{papadopoulos-1963}
\begin{equation}
  B=\frac{\mu_0IR^2}{2(R^2+z^2)^{\frac{3}{2}}}
\end{equation}

Se remplazan los términos de \ref{v=E/B} en \ref{v=sqrt} dejando la expresión
\begin{equation}
  \frac{E}{B}=\sqrt{\frac{2eU_a}{m_e}}
\end{equation}

El campo eléctrico $E$ puede ser remplazado por la ecuación $E=\frac{U_D}{d}$
debido a que es el campo de un capacitor en donde $d$ es la distancia entre las
placas, dejando la expresión
\begin{equation}\label{almost e/m}
  \frac{U_D}{Bd}=\sqrt{\frac{2eU_a}{m_e}}
\end{equation}

Despejando la ecuación \ref{almost e/m} se llega a la carga específica del
electrón
\begin{equation}
  \frac{e}{m_e}=\frac{U_D^2}{2u_a}\frac{1}{B^2d^2}
\end{equation}
