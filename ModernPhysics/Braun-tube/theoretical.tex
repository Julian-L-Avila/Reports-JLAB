Para determinar la carga específica del electrón, se aceleran electrones mediante
una diferencia de potencial, de modo que la energía cinética adquirida es
equivalente a la energía potencial eléctrica, en concordancia con el principio
de conservación de la energía.
Esta relación se expresa en la \cref{eq:electron-energy}.
\begin{equation}\label{eq:electron-energy}
  e U_a =\frac{1}{2} m_e v^{2}_{x}
\end{equation}

Donde \(U_a\) es el potencial de aceleración, \(e\) la carga del electrón,
\(m_e\) su masa y \(v_x\) la velocidad horizontal alcanzada.
La velocidad horizontal puede despejarse para obtener la
\cref{eq:velocity-electric}.
\begin{equation}\label{eq:velocity-electric}
  v_x = \sqrt{\frac{2 e U_a}{m_e}}
\end{equation}

Al aplicar un potencial de deflexión, el electrón adquiere una componente de
velocidad vertical, lo que provoca un desplazamiento en su posición final sobre
la pantalla.
Para contrarrestar dicha deflexión, se aplica un campo magnético.

Cuando la fuerza eléctrica generada por el potencial de deflexión y la fuerza de
Lorentz debida al campo magnético están en equilibrio, se establece la relación
de velocidad que se presenta en la \cref{eq:veloctiy-relation}, donde \(U_d\) es
el potencial de deflexión y \(d\) la distancia entre placas del capacitor.
\begin{equation*}
  e \frac{U_d}{d} = e v_x B
\end{equation*}
\begin{equation}\label{eq:veloctiy-relation}
  v_x = \frac{U_d}{Bd}
\end{equation}

El campo magnético, generado por un par de bobinas de Helmholtz,
puede calcularse utilizando la \cref{eq:magnetic-field} \cite{jackson-1998},
donde \(n = 240 \) el número de vueltas de las bobinas y \(r = \qty{11.2e-2}{m}\)
es su radio.
\begin{equation}\label{eq:magnetic-field}
  B = \left( \frac{4}{5} \right)^{\frac{3}{2}} \frac{\mu_0 n I}{R}
\end{equation}

Al combinar las dos expresiones para la velocidad horizontal, dadas por las
\cref{eq:velocity-electric,eq:veloctiy-relation}, se obtiene la
\cref{eq:specific-charge},que permite determinar explícitamente la carga
específica del electrón.
\begin{equation}\label{eq:specific-charge}
  \frac{e}{m_e}=\frac{1}{2U_a}\frac{U^2_d}{B^2d^2}
\end{equation}
