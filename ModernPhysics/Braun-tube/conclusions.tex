Se logró medir tres diferentes potenciales de aceleración, y para cada uno de ellos se determinó el potencial de deflexión necesario 
para alinear el haz en los tres puntos utilizados con respecto al punto de referencia. Se evidenció que, a mayor potencial de aceleración,
es necesario un mayor potencial de deflexión para que el haz choque con el punto deseado. Por otro lado, se determinó la carga específica del 
electrón, obteniendo un valor de \(e / m \approx \qty{1.2(5)e+12}{C \per kg}\) con un error relativo porcentual del \qty{2.98}{\percent} 
utilizando el límite inferior del intervalo de confianza. Se realizó un análisis del campo magnético obtenido y se determinó que los valores 
teóricos del campo magnético difieren casi en un orden de magnitud de los valores experimentales, lo que sugiere que una forma de mejorar la 
metodología empleada es medir directamente el campo eléctrico generado por las bobinas.
