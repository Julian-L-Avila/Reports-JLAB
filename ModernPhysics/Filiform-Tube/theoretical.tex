Para determinar la carga específica del electrón, se pone a un electrón que se mueve a una velocidad $v$ bajo el campo magnético 
generado por un par de Bobinas de Helmholtz, dichas bobinas representan una de las configuraciones más
simples para generar un campo magnético relativamente constante. Estas bobinas consisten en dos bobinas circulares coaxiales de igual
radio, el cual es igual a la distancia entre los planos de las bobinas.



\begin{figure}[H]
    \centering
    \includegraphics[width=0.8\linewidth]{images/Campo de una espira (1).pdf}
    \caption{Campo magnético de una espira.}
    \label{fig:una_espira}
\end{figure}

Para calcular el campo magnético generado por estas bobinas, primero se determina el campo producido por una de ellas. En este caso, se considera una espira por la que circula una corriente eléctrica constante $I$ en el tiempo. Partimos de la expresión diferencial de la ley de Biot-Savart:

\begin{equation}\label{eq:Biot-Savart}
    d\mathbf{B}(\mathbf{r}) = \frac{\mu_0 I}{4 \pi} \frac{d\mathbf{l} \times \mathbf{R}}{R^{3}}
\end{equation}

Donde:

$d\mathbf{l}=a d\phi \hat{\phi}$ 

$\mathbf{R}=\mathbf{r}-\mathbf{r'}$ 

$\mathbf{r}=z\hat{e_{z}}$

$\mathbf{r'}=a\hat{e_{\rho}}$

$R=\sqrt{z^{2}+a^{2}}$ 

Por lo que reemplazando se tiene que:

\begin{equation} 
    d\mathbf{B}(\mathbf{r}) = \frac{\mu_{0}I}{4\pi} \frac{(a d\phi z \hat{\rho}) + (a^{2}d\phi \hat{z})}{(z^{2}+a^{2})^{3/2}}
\end{equation} 

Al integrar se obtiene el valor de $\mathbf{B}(\mathbf{r})$ y por la simetría del problema se evidencia que $\mathbf{B}(\mathbf{r}) = B(r)\hat{z}$, por lo que se obtiene:

\begin{equation} 
	\mathbf{B}(\mathbf{r}) = \frac{\mu_{0}I}{4 \pi} \frac{a^{2}}{(z^{2}+a^{2})^{3/2}} \hat{e_{z}} \int _{0}^{2 \pi} d\phi
\end{equation}

Lo que da como resultado:

\begin{equation} 
    \mathbf{B}(\mathbf{r}) = \frac{\mu_{0}I}{2} \frac{a^{2}}{(z^{2}+a^{2})^{3/2}} \hat{e}_{z}
    \label{eq:Campo_magnetico_espira}
\end{equation} 

\begin{figure}[H]
    \centering
    \begin{subfigure}[a]{0.45\textwidth}
        \centering
        \includegraphics[width=0.6\textwidth]{images/dos espiras (1).pdf}
        \caption{Corriente en dos espiras}
        \label{fig:I_en_dos_espiras}
    \end{subfigure}
    \hfill
    \begin{subfigure}[b]{0.45\textwidth}
        \centering
        \includegraphics[width=0.6\textwidth]{images/Campo de bobinas (1).pdf}
        \caption{Campo magnético entre dos espiras circulares}
        \label{fig:CM_dos_espiras}
    \end{subfigure}
    \caption{Dos espiras circulares}
    \label{fig:dos_espiras}
\end{figure}

El campo magnético $\mathbf{B}$ producido por dos espiras, como se observa en la \cref{fig:dos_espiras}, donde circula una corriente estática $I$ en cada una de las espiras, se calcula sumando dos componentes del vector $\mathbf{B}$ de acuerdo al resultado de la \cref{eq:Campo_magnetico_espira}. Como resultado, se obtendrán dos términos iguales, considerando que las espiras se encuentran localizadas en $z = -d/2$ y $z = +d/2$.

\begin{equation}
\begin{aligned}
    \mathbf{B}_{z}(\rho=0,z) &= \frac{\mu_{0} I a^{2}}{2} \left[ \frac{1}{\left((z-\frac{d}{2})^{2} + a^{2}\right)^{3/2}} \right.\\
    &\quad \left.+ \frac{1}{\left((z+\frac{d}{2})^{2} + a^{2}\right)^{3/2}} \right] \hat{e}_{z}
    \label{Campo_magnetico_2espiras}
\end{aligned}
\end{equation}

Para el caso del montaje experimental, las bobinas tienen una separación igual a su radio, por lo que \( d = a \). Además, el haz de electrones se ubica en el centro de las bobinas, es decir, \( z = 0 \). Teniendo en cuenta todas estas consideraciones, se llega a que el campo magnético producido por las bobinas en el haz de electrones es \cite{boix_practica_2}:

\begin{equation}
    \mathbf{B}_{z}(\rho=0, z=0) = \frac{8 n \mu_{0} I}{5 \sqrt{5} R}
    \label{eq:Campo_magnetico_enHaz}
\end{equation}

donde \( R = a \).

\begin{figure}[H]
    \centering
    \includegraphics[width=0.8\linewidth]{images/Mov-electron (1).pdf}
    \caption{Trayectoria del haz de electrones}
    \label{fig:haz_elelctrones}
\end{figure}

Dado que el electrón, con carga \( e \), se mueve perpendicularmente al campo magnético, está sujeto a la fuerza de Lorentz, cuya magnitud está dada por:

\begin{equation}
    F = e v B
    \label{eq:FuerzaLorentz}
\end{equation}

Debido a que la fuerza de Lorentz es perpendicular a la velocidad y al campo magnético, genera un movimiento circular en el haz de electrones, por lo que también actúa como una fuerza centrípeta:

\begin{equation}
    F = m_{e} \frac{v^{2}}{r}
    \label{eq:Fuerzacentripeta}
\end{equation}

Igualando las \cref{eq:FuerzaLorentz,eq:Fuerzacentripeta}, se obtiene una relación carga-masa en función de la velocidad, el campo magnético y el radio del haz de electrones:

\begin{equation}
    \frac{e}{m_{e}} = \frac{v}{r B}
    \label{eq:Primera_carga-masa}
\end{equation}

Sin embargo, la velocidad con la que viaja el haz de electrones es desconocida. Para determinarla, se recurre a la conservación de la energía. En el experimento, los electrones son acelerados en un tubo de rayo electrónico filiforme por una diferencia de potencial \( U \), de modo que la energía eléctrica se transforma en energía cinética:

\begin{equation}
\begin{aligned}
    e U &= \frac{1}{2} m_{e} v^{2} \\
    \Rightarrow \frac{e}{m_{e}} &= \frac{v^{2}}{2 U}
\end{aligned}
\label{eq:Segunda_carga-masa}
\end{equation}

Igualando las ecuaciones \cref{eq:Primera_carga-masa,eq:Segunda_carga-masa}, se obtiene que la velocidad es \( v = \frac{2 U}{r B} \), y finalmente, se llega a que la carga específica del electrón es: \cite{leybold_experimentos}

\begin{equation}
    \frac{e}{m_{e}} = \frac{2 U}{(r B)^{2}}
    \label{eq:carga-masa}
\end{equation}







