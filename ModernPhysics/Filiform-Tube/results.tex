Utilizando el montaje descrito, se obtuvieron 22 conjuntos de datos
correspondientes a cinco potenciales de aceleración diferentes.
Con base en la \cref{eq:Campo_magnetico_enHaz}, se calculó el valor del campo
magnético generado por las bobinas de Helmholtz, mientras que con la
\cref{eq:carga-masa} se determinó la carga específica del electrón para cada caso.
Los resultados obtenidos se presentan en la \cref{tab:data}.

\begin{table}[htbp!]
	\centering
	\rowcolors{2}{white}{gray!25}
	\begin{tabular}{S[table-format=3.0(1)]|S[table-format=1.2(1)]|S[table-format=2.1(1)]|S[table-format=1.2(1)]|S[table-format=1.2(1)]}
		\toprule
		{$U \, (\unit{V})$} & {$I \, (\unit{A})$} & {$r \, (\unit{\centi m})$} &
			{$B \, (\unit{\milli T})$} & {$e/m \, (\unit{\tera C \per kg})$} \\
		\midrule
		153(1) & 1.03(1) & 08.8(1) & 0.60(1) & 0.11(4) \\
		153(1) & 1.52(1) & 07.9(1) & 0.89(1) & 0.06(2) \\
		153(1) & 2.00(1) & 05.5(1) & 1.17(1) & 0.07(2) \\
		153(1) & 2.52(1) & 04.5(1) & 1.47(1) & 0.06(2) \\
		201(1) & 1.03(1) & 09.2(1) & 0.60(1) & 0.13(5) \\
		201(1) & 1.51(1) & 06.3(1) & 0.88(1) & 0.13(5) \\
		201(1) & 2.00(1) & 04.9(1) & 1.17(1) & 0.12(4) \\
		201(1) & 2.53(1) & 04.0(1) & 1.48(1) & 0.11(4) \\
		250(1) & 1.02(1) & 10.4(1) & 0.60(1) & 0.13(4) \\
		250(1) & 1.51(1) & 07.0(1) & 0.88(1) & 0.13(5) \\
		250(1) & 2.01(1) & 05.2(1) & 1.17(1) & 0.13(5) \\
		250(1) & 2.50(1) & 04.2(1) & 1.46(1) & 0.13(5) \\
		300(1) & 1.02(1) & 10.4(1) & 0.60(1) & 0.15(5) \\
		300(1) & 1.50(1) & 08.0(1) & 0.88(1) & 0.12(4) \\
		300(1) & 2.03(1) & 05.8(1) & 1.19(1) & 0.12(4) \\
		300(1) & 2.50(1) & 05.0(1) & 1.46(1) & 0.11(3) \\
		300(1) & 3.00(1) & 04.7(1) & 1.75(1) & 0.08(2) \\
		350(1) & 1.02(1) & 12.5(1) & 0.60(1) & 0.12(3) \\
		350(1) & 1.48(1) & 08.8(1) & 0.87(1) & 0.12(3) \\
		350(1) & 2.02(1) & 06.6(1) & 1.18(1) & 0.11(3) \\
		350(1) & 2.51(1) & 05.3(1) & 1.47(1) & 0.11(3) \\
		350(1) & 3.03(1) & 04.5(1) & 1.77(1) & 0.11(3) \\
		\bottomrule
	\end{tabular}
	\caption{Valores para el campo magnético y la carga especifica.}
	\label{tab:data}
\end{table}

Los datos revelan una tendencia coherente con las predicciones teóricas,
permitiendo calcular un valor promedio para la relación carga-masa del electrón
de \(e/m =\qty{0.12(4)}{\tera C \per kg}\).
Al comparar este valor con el valor aceptado en la literatura de
\(e/m =\qty{0.176}{\tera C \per kg}\), se observó un error relativo porcentual
del \qty{9.09}{\percent}, considerando el límite superior del intervalo de
confianza.

La varianza de los datos, calculada en \num{0.229}, indica una dispersión
moderada en las mediciones, posiblemente atribuible a fluctuaciones en la
precisión instrumental y a la estabilidad del campo magnético.
A pesar de esta dispersión, el valor promedio obtenido es razonablemente cercano
al valor teórico, lo que sugiere que la metodología es adecuada para una
estimación aproximada de la relación carga-masa del electrón.

Sin embargo, para mejorar la precisión de los resultados, sería óptimo medir el
campo magnético de manera directa, en lugar de calcularlo a partir de la
corriente en las bobinas. Esto reduciría las posibles inexactitudes introducidas
por la dependencia del campo en factores como la geometría del sistema y la
homogeneidad del campo generado.
