Se logró determinar la carga específica del electrón, obteniendo un valor
promedio de \( e/m = \qty{0.12(4)}{\tera\C\per\kg} \), con un error del
\qty{9.09}{\percent} en comparación con los datos de la literatura usando el
limite superior del intervalo de confianza, y un error del \qty{31}{\percent}.
Además, se determinó el campo magnético generado por las bobinas para diferentes
valores de corriente.
Se observó que a mayor corriente, menor era el radio de la órbita, por que se
deduce un aumento en la fuerza magnética y el campo que la produce.
