En conclusión, el experimento de rayos catódicos ha demostrado la efectividad de los métodos experimentales utilizados para investigar
 las propiedades de las partículas subatómicas. La combinación de campos magnéticos resultó ser una herramienta poderosa para observar 
 cómo los rayos catódicos se desvían en presencia de estos campos, subrayando la importancia de estos elementos en el análisis del 
 comportamiento de las partículas cargadas.\\

La capacidad de medir con precisión la desviación de los rayos catódicos en respuesta a diferentes intensidades de campos confirma
 la necesidad de un control riguroso y exacto en la aplicación de las condiciones experimentales. Esta precisión es fundamental para 
 obtener datos confiables y reproducibles en estudios de física.\\

El experimento también destaca la relevancia de aplicar técnicas modernas para el estudio de las partículas subatómicas y su comportamiento
 bajo diversas condiciones. El proceso y la metodología empleada ofrecen una base sólida para futuras investigaciones en el campo de la física 
 de partículas.\\

Finalmente, la experiencia adquirida en el diseño y ejecución del experimento contribuye a una comprensión más profunda de los principios
 experimentales en la física. Las lecciones aprendidas en este contexto pueden ser aplicadas para mejorar y optimizar métodos experimentales 
 en investigaciones futuras, avanzando así en el conocimiento científico en esta área.
