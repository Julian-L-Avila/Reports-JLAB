\selectlanguage{english}
\begin{abstract}
  The specific charge of the electron was determined using an electron gun and a
  pair of Helmholtz coils to generate a uniform magnetic field.
  The electron gun was supplied with voltages ranging from \num{150} and \qty{350}{V},
  while the current through the coils was varied between \num{1} and \qty{3}{A}.
  The diameter of the circular trajectory of the electron beam was measured to
  calculate the magnetic field and subsequently the specific charge of the
  electron.
  The results yielded a value of \(e = \qty{0.12(4)}{\tera C \per kg}\), with a
  relative error of \qty{9.09}{\percent} when considering the upper limit of the
  confidence interval and \qty{31}{\percent} when using the mean value, compared
  to the accepted theoretical value.
  The need for direct measurement of the magnetic field is emphasised to improve
  the accuracy of the results.
\end{abstract}
\begin{IEEEkeywords}
  Specific charge; Helmholtz's Coils, Electron Gun.
\end{IEEEkeywords}

\selectlanguage{spanish}
\begin{abstract}
  Se determinó la carga específica del electrón utilizando un cañón de electrones
  y un par de bobinas de Helmholtz para generar un campo magnético uniforme.
  El cañón fue alimentado con voltajes entre \num{150} y \qty{350}{V},
  mientras que la corriente en las bobinas se varió entre \num{1} y \qty{3}{A}.
  Se midió el diámetro de la trayectoria circular del rayo de electrones para
  calcular el campo magnético y, posteriormente, la carga específica del electrón.
  Los resultados mostraron un valor \(e = \qty{0.12(4)}{\tera C \per kg}\),
  con un error relativo del \qty{9.09}{\percent} al considerar el límite superior
  del intervalo de confianza y del \qty{31}{\percent} al tomar el valor medio,
  en comparación con el valor teórico aceptado.
  Se subraya la necesidad de realizar mediciones directas del campo magnético para
  mejorar la precisión de los resultados.
\end{abstract}

\begin{IEEEkeywords}
  Carga específica, Bobinas de Helmholtz, Canion de electrones.
\end{IEEEkeywords}
